\documentclass[11pt, preprint]{aastex}

\newcommand{\todo}[1]{{\bf #1}}

\newif\ifanswers


\begin{document}

\title{\bf Computational Physics / PHYS-UA 210 / Problem Set \#1
\\ Due September 6, 2019 }

Submit this homework to the TA as a link to the Jupyter Notebook and
(for \#2) a PDF file, checked into your GitHub account, in the place
specified in the class notes for Lecture \#1.

You {\it must} label all axes of all plots, including giving the {\it
  units}!!

\begin{enumerate}

  \item Familiarize yourself enough with a plotting package in Python
    (e.g. {\tt matplotlib}) to plot a Gaussian with zero mean and a
    standard deviation of 3 over the range [$-$10, $+$10]. Make sure
    the Gaussian is normalized correctly.

  \item Create a short LaTeX document. You may use code on your laptop
    or you may use Overleaf or ShareLatex. For material for this
    describe your goals for this course, your background in
    programming and/or numerics, and (to the extent you know them)
    your plans after your degree is finished (grad school? industry?
    law school? etc.). Write just one paragraph! If you write more
    than a page, you have written way too much. You won't be graded on
    the content of this, just whether you do it!! It will also help me
    understand what you want out of the class.

\end{enumerate}

\end{document}
