\documentclass[11pt, preprint]{aastex}

\newcommand{\todo}[1]{{\bf #1}}

\newif\ifanswers


\begin{document}

\title{\bf Computational Physics / PHYS-UA 210 / Problem Set \#5
\\ Due October 13, 2017 }

You {\it must} label all axes of all plots, including giving the {\it
  units}!!

\begin{enumerate}
  \item Exercise 3 in S5.16 of Landau.
  \item Exercise 4 in S5.16 of Landau.
  \item A common type of multidimensional integral I need to do is as
    follows. We observe galaxy images focused on a 2D telescope focal
    plane, using optical fibers in the focal plane that carry the
    light to a bank of spectrographs. Each fiber has a circular
    aperture in the focal plane, so the light detected from the galaxy
    is that within some radius of its center. Two pretty good models
    for some galaxies are the following (the exponential, and de
    Vaucouleurs models): 
    \begin{eqnarray}
     I_{\rm exp}(r) = A \exp\left[- \left(1.678 r/r_{50}\right)\right] \cr
     I_{\rm deV}(r) = A \exp\left[- \left(3466 r/r_{50}\right)^{1/4}\right]
    \end{eqnarray}
    The galaxy is not necessarily azimuthally symmetric, it can appear
    ``squashed'' on the sky, due to our viewing it at an angle. This
    can be accounted for by saying:
    \begin{equation}
      r = \sqrt{x^2 + \left(\frac{a}{b} \right)^2 y^2}
    \end{equation}
    where $b/a < 1$ is called the {\it axis ratio}. Alternatively, one
    can express this as:
    \begin{eqnarray}
      r = r' \sqrt{\cos^2\theta +
        \left(\frac{a}{b}\right)^2\sin^2\theta} \cr
      r' = \sqrt{x^2 + y^2}
    \end{eqnarray}
    where $\tan \theta = y/x$. 
    
   Using $r_{50}$ = 2 arcseconds, and assuming a fiber radius of 1
   arcsecond, calculate the fraction of the light that goes into the
   fiber as a function of $b/a$ for each model, within the range $0.1$
   to $1$. First, think about whether you want to integrate over $x$
   and $y$ or over $r'$ and $\theta$. Check your convergence carefully
   for $n=4$!
\end{enumerate}

\end{document}
