\documentclass[11pt, preprint]{aastex}

\newcommand{\todo}[1]{{\bf #1}}

\newif\ifanswers


\begin{document}

\title{\bf Computational Physics Project / Modeling a point spread
  function}

This project involves fitting a model to observational data, to
charactering the resolution of an image from a telescope.

Telescope images have a resolution that changes across their field of
view, because of their optics and because of the atmosphere.  Precise
measurements of these images require first determining these {\it
  point spread functions} as a function of image position. This point
spread function can be thought of as a Green's Function --- i.e. the
response of the system to a delta-function input.

Here you will fit a simple model to the point spread function of
images. 

You can retrieve images at the
\href{http://dr12.sdss.org/fields}{\color{blue} SDSS imaging data
  site}. You can use whatever fields you want, but here are some
examples with ``run,'' ``camcol,'' and ``field'' values:
\begin{verbatim}
94 3 200
752 4 400
1336 6 61
\end{verbatim}
I'll ask you to look at all the bandpasses for whatever fields you
choose.

You will need the FITS format files. These files can be read in using
the {\tt fitsio} Python package, which is {\tt pip}-installable. You
will be able to read the images into a NumPy array with:
\begin{verbatim}
image = fitsio.read(filename, ext=0)
\end{verbatim}
In order to look at the image within Python, you will need to use the
{\tt imshow} utility in {\tt matplotlib}.  Note that these images
already are calibrated and background-subtracted; they don't come off
the telescope looking this clean!

Some of the sources in the image are stars, and some are
galaxies. Stars are ``unresolved,'' in the sense that how big the
image appears is set purely by the point spread function. Their
intrinsic angular width is extremely small. Galaxies are generally
somewhat bigger on the sky than stars --- they are often ``resolved.''
The result is that in order to determine the point spread function,
you should limit yourself to stars. 

For any given field, you may consult the
\href{http://skyserver.sdss.org/dr14/en/tools/search/sql.aspx}{\color{blue}
  SQL query server on SkyServer} to get a list of identified stars in
it, using the following query:
\begin{verbatim}
SELECT colc_u,colc_g,colc_r,colc_i,colc_z,
       rowc_u,rowc_g,rowc_r,rowc_i,rowc_z
FROM photoObj
WHERE run = 94 AND camcol = 2 AND field = 200 AND type = 6
\end{verbatim}
This will return the row and column positions (defined with 0 at the
first pixel edge, not the center). Use the ``CSV'' or ``FITS'' output
formats for ease of reading the results into Python.

\begin{itemize}
\item Download the image and the star positions for at least one field
  and bandpass. The star positions over a section of the image, to
  verify you are interpreting the row and column values correctly.
\item Write code to: identify isolated stars (stars without any other
  nearby star within 20 pixels); cut out 31$\times$31 pixels around
  each; shift each array so that the stellar center is the center of a
  pixel using a high order 2-dimensional interpolation; normalize each
  image using the flux with 15 pixels radius of the center. 
\item Apply principal components analysis (PCA; Landau Chapter 13.7)
  on these images. Do not forget to subtract off the mean before
  applying PCA. Perform an analysis of the residual error as a
  function of number of components retained.
\item Probably 3--4 components are okay.  Write a routine to fit a
  low-order (i.e. up to 2 or so periods across the image) sum of sine
  and cosine functions to each PCA coefficient in the row and column
  directions, using least squares on the values of the coefficient for
  each star. Use SVD for this solution for stability. E.g. your model
  for coefficient $i$ from the PCA will look like:
  \begin{equation}
    a_i  = \sum_j A_j \sin(2\pi j x / n_x) + B_j \cos(2\pi j y / n_y)
  \end{equation}
  in terms of the linear model parameters $A_j$ and $B_j$.
\item The PSF at any location can be recovered from your fit to the
  $a_i$. Test the goodness-of-fit by subtracting the resulting model from
  the stars. 
\item Run your code on all the bandpasses for several images. 
\end{itemize}

\end{document}
