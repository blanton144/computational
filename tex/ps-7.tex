\documentclass[11pt, preprint]{aastex}

\newcommand{\todo}[1]{{\bf #1}}

\newif\ifanswers


\begin{document}

\title{\bf Computational Physics / PHYS-UA 210 / Problem Set \#7
\\ Due October 27, 2017 }

You {\it must} label all axes of all plots, including giving the {\it
  units}!!

Here you will solve the heat equation using the same techniques used
in class for the vibrating string. This equation describes a
heat-conducting rod, with in this case insulated ends.

The equation in one dimension is:
\begin{equation}
  \frac{\partial u(x, t)}{\partial t} -
  \alpha \frac{\partial^2 u(x,t)}{\partial x^2} = 0
\end{equation}
and if its ends are insulated it is subject to the boundary condition:
\begin{equation}
  \frac{\partial u(x, t)}{\partial x} = 0
\end{equation}
Constraints on the derivative at the boundary are known as {\it
  Neumann} boundary conditions (constraints on the function value at
the boundary are known as {\it Dirichlet} boundary conditions.

\begin{enumerate}
\item Use the method of separation of variables to analytically find
  the solutions of this equation for constant $\alpha$. 
\item Now use the finite difference method presented in class to solve
  the same problem using a finite eigensystem. Compare the
  eigenfunctions to your analytic method and test how its accuracy
  varies with $N$. Demonstrate how a central temperature excess (use a
  Gaussian with a standard deviation of about one pixel) evolves over
  time. Feel free to use an altered version of the {\tt StringProb}
  class in the Jupyter notebook for this lecture. However, you will
  need to account for the boundary conditions differently. To do so,
  use a central difference approximation to the derivative at the
  first node, and use this to alter the first and last finite
  difference equations.
\item Now alter the thermal diffusivity coefficient $\alpha$ to be a
  function of position. Try putting a ``barrier'' of low diffusivity
  somewhere (not right at the location of the Gaussian), and see what
  happens. Try some other pattern you invent too.
\end{enumerate}

\end{document}
