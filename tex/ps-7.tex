\documentclass[11pt, preprint]{aastex}

\newcommand{\todo}[1]{{\bf #1}}

\newif\ifanswers


\begin{document}

\title{\bf Computational Physics / PHYS-UA 210 / Problem Set \#7
\\ Due October 18, 2019 }

You {\it must} label all axes of all plots, including giving the {\it
  units}!!

\begin{enumerate}
\item Exercise 6.2 in Newman.
\item Consider Example 6.2 in Newman. We will alter this problem to
  handle a heterogeneous set of masses.
  \begin{enumerate}
    \item Rewrite Equation 6.56 with a heterogeneous set of masses
      $m_i$. 
    \item Alter the code in {\tt springs.py} to use a heterogeneous
      set of masses. Test it for constant mass $m_i = 1$ and
      demonstrate that it gets the same results as the unaltered code.
    \item Test putting a large mass near the middle, $m_{13} = 10$,
      with $m_i = 1$ otherwise. 
    \item Test putting a small mass near the middle $m_{13} = 0.1$.
  \end{enumerate}
\item We will further consider Example 6.2 in Newman, now altering it
  to account for dissipation. 
  \begin{enumerate}
    \item Consider the case that there is a dissipative term on the
      RHS of Equation 6.50 with an amplitude $-\gamma \dot\xi_i$.
      Alter the code in {\tt springs.py} so that it uses the {\tt inv}
      function in {\tt numpy.linalg} instead of performing the inverse
      itself.
    \item How does $x_i$ vary with position and $\gamma$?
  \end{enumerate}
\end{enumerate}

\end{document}
