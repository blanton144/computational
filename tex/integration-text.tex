\section{Basic idea of integration}

Integration of functions is an important calculation in computational
physics, both as a fundamental task and as a component of larger
problems. Many integrals do not have closed forms and require
numerical computation.

\noindent {\bf What is the definition of an integral?}

\begin{answer}
An integral is defined by the limit:
\begin{equation}
\int_a^b {\rm d}x f(x) = \lim_{{\rm d}x \rightarrow 0} \left[{\rm d}x
  \sum_{i=1}^{(b-a)/{\rm d}x} f(x_i) \right]
\end{equation}
where $x_i$ are spaced between $a$ and $b$ with separations ${\rm
  d}x$.
\end{answer}

\noindent {\bf What is a simple numerical estimate of an integral?}

\begin{answer}
Just to perform this sum with some finite ${\rm d}x$:
\begin{equation}
\int_a^b {\rm d}x f(x) = \left[{\rm d}x \sum_{i=1}^{(b-a)/{\rm d}x}
  f(x_i) \right]
\end{equation}
where $x_i$ are spaced between $a$ and $b$ with separations ${\rm
  d}x$.
\end{answer}

This is just a particular case of the more general form that most
integration methods take, which is that it can be approximated as some
linear combination of evaluations of the function:
\begin{equation}
  \int_a^b {\rm d}x f(x) = \sum_{i=1}^N 
  f(x_i) w_i
\end{equation}

\section{Trapezoid rule}

The simple estimate above can be thought of as approximating the
function as piecewise constant. Obviously there are better
approximations that can be made! Better algorithms for integration
generally boil down to better models of the function. In this respect,
integration is closely allied to interpolation of functions.

The trapezoid rule is the result of integrating a linear interpolation
of the function. Each term in the integral will become:
\begin{equation}
\frac{1}{2} {\rm d}x \left( f_i + f_{i+1} \right) 
\end{equation}
The next term is:
\begin{equation}
\frac{1}{2} {\rm d}x \left( f_{i+1} + f_{i+2} \right) 
\end{equation}
For equally spaced points, then $w_i = {\rm d}x$, except for $w_1=
w_{N} = {\rm d}x/2$.

\noindent {\bf For what sort of function is the trapezoid rule exactly
  correct?} 

\begin{answer}
For a linear function. Of course, this property is not very useful!!
\end{answer}

\section{Simpson's rule}

Simpson's rule represents the next level of sophistication in
interpolation. Here, the function is approximated locally around the
points $i-1$, $i$, $i+1$, as a quadratic:
\begin{equation}
f(x) = \alpha' + \beta' x + \gamma' x^2
\end{equation}
This is not a very convenient form. Let us instead use:
\begin{equation}
  f(x) = \alpha + \beta \left(\frac{x - x_i}{{\rm d}x}\right) +
  \gamma \left(\frac{x - x_i}{{\rm d}x}\right)^2 = 
  \alpha + \beta y
  \gamma y^2
\end{equation}
with a change of variable to $y = (x-x_i)/{\rm d}x$.  For a set of
three points, $i-1$, $i$, and $i+1$, you can fit the parabola using
the fact:
\begin{eqnarray}
f_{i-1} &=& \alpha - \beta + \gamma \cr
f_{i} &=& \alpha \cr
f_{i+1} &=& \alpha + \beta + \gamma
\end{eqnarray}
This can be easily solved:
\begin{eqnarray}
\alpha &=& f_i \cr
\gamma &=& \frac{f_{i+1}+f_{i-1}}{2} - f_i \cr
\beta &=& \frac{f_{i+1}-f_{i-1}}{2}
\end{eqnarray}

\noindent {\bf What is the integral over the region defined by these
  three points?}

\begin{answer}
The integral over the region defined by these three points :
\begin{eqnarray}
  \int_{x_{i-1}}^{x_{i+1}} {\rm d}x f(x) &=& {\rm d}x \int_{-1}^{1} {\rm
    d}y \left(\alpha + \beta y + \gamma y^2\right) \cr
  &=& {\rm d}x \left[\alpha y + \frac{\beta}{2} y^2 + \frac{\gamma}{3}
    y^3\right]_{-1}^{1}  \cr
  &=& {\rm d}x \left[2 \alpha + \frac{2\gamma}{3}\right]
\end{eqnarray}
Plugging in $\alpha$ and $\gamma$:
\begin{equation}
  \int_{x_{i-1}}^{x_{i+1}} {\rm d}x f(x)
  = {\rm d}x \left(2 f_i +
  \frac{f_{i+1} + f_{i-1}}{3} - \frac{2}{3} f_i\right) 
  = {\rm d}x \left(\frac{1}{3} f_{i-1}
  + \frac{4}{3} f_i 
  + \frac{1}{3} f_{i+1}\right)
\end{equation}
\end{answer}

Simpson's rule comes from using this approximation across the length
from $a$ to $b$, by dividing the interval into an even number of
segments, and integrating each separately. This yields a full
summation:
\begin{equation}
  \int_a^b {\rm d}x f(x) = \sum_{i=1}^N = {\rm d}x \left[\frac{1}{3}
    f_1 + \frac{4}{3} f_2 + \frac{2}{3} f_3 + \frac{4}{3} f_4 + \ldots
    + \frac{2}{3} f_{N-2} + \frac{4}{3} f_{N-1} + \frac{1}{3} f_N
    \right]
\end{equation}

Because this is applied to two segments at a time, it requires an even
number of segments, which means $N$ must be odd.

\section{Gaussian quadrature}

\section{Romberg Integration}

\section{Dealing with singularities and infinities}

\section{NumPy implementation}
