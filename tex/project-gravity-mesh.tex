\documentclass[11pt, preprint]{aastex}

\newcommand{\todo}[1]{{\bf #1}}

\newif\ifanswers


\begin{document}

\title{\bf Computational Physics Project / Gravity on a mesh}

This project focuses on solving Poisson's equation on a uniform mesh.
The problem you will solve here is {\it one piece} of a larger
technique known as {\it particle-mesh} which is often used in the
context of simulating the formation of structure in the universe and
in other contexts. 

In order to follow how structure forms due to gravity, the
particle-mesh technique approximates the distribution of matter using
a large set of particles in 3-dimensions. The simulation starts at
some initial time step with some initial conditions for the
particles. The density field is approximated by counting the number of
particles in each cell of a rectangular mesh. This yields a 3D array
containing the density field. Then, Poisson's equation is solved for
the gravitational potential. The acceleration on each particle can
then be calculated as the gradient of the potential. The velocity of
the particle is updated, and its position is updated according to the
velocity. Then the process is repeated with the new (slightly
different) density field. The basics of this method were described
thoroughly in a classic book by Hockney \& Eastwood (1988), {\it
  Computer Simulation Using Particles}.

In this project, you will just perform and test the solution of
Poisson's equation for a density field that you define. Then you will
play with integrating orbits in this potential (without updating the
potential). Chapter 19 of Landau has some information about this;
however, in the gravitational case you cannot do exactly what it says
there.

\section{Defining a density field}

Your first task will be to define a 3D density field. For this
purpose, you should use a multivariate Gaussian distribution. That is,
the density field will have the form:
\begin{equation}
\rho(\vec{x}) = \left(\frac{1}{\sqrt{2\pi} \det{\bf C}}\right)^{1/2}
\exp\left(-\frac{(\vec{x}-\vec{x}_0)\cdot{\bf
    C}^{-1}\cdot(\vec{x}-\vec{x}_0}{2}\right)
\end{equation}
where ${\bf C}$ is the {\it covariance matrix} of the Gaussian and
$\vec{x}_0$ is the location of the peak of the Gaussian. It is
diagonal with equal elements for a spherically symmetric Gaussian.

\begin{itemize}
\item Write a routine to create a mesh with such a density field
  taking the covariance, center, and the size of the mesh as
  input. You should put this routine into a module that you can call.
\item Then write a routine to plot slices of the mesh, plus any other
  sort of visualization that you want to explore, to show that the
  Gaussian is behaving as you expect in the following cases.
\item Experiment with different axis ratios for the Gaussian by
  changing the diagonal elements.
\item Experiment with using principal axes of the Gaussian not aligned
  with the mesh. Note you should {\it rotate} the covariance matrix
  elements; just making up random off-diagonal elements will not
  necessarily behave as you expect.
\end{itemize}

\section{Solving Poisson's equation}

Then you need to solve Poisson's equation:
\begin{equation}
\nabla^2 \phi(\vec{x}) = - 4\pi \rho(\vec{x})
\end{equation}
You should use Fourier methods to perform the solution --- i.e. use
the fact that the Fourier transform of the above equation is:
\begin{equation}
k^2 \tilde\phi(\vec{k}) = - 4\pi \tilde\rho(\vec{k})
\end{equation}

A very similar case is described in the book.  However, unlike the
electrostatics cases described in the book, you do not have boundary
conditions of $\phi$ at the boundary. Instead, you need to use a
trick to treat the mass distribution on the mesh as
isolated. 

\begin{itemize}
\item Write a function in your module to solve Poisson's equation
  using this method. 
\item Test this method by making a relatively small spherical,
  Gaussian making sure that at large distances you get the Keplerian
  potential you expect.
\item Test this method further by making a bigger spherical Gaussian
  and making sure that even where the density is high you get the
  potential you expect. 
\item Now explore a prolate or oblate Gaussian. Use a fairly extreme
  axis ratio (e.g. 0.1) and compare the density and the potential.
  Does the potential  have the same axis ratio?
\end{itemize}

\section{Bonus: integrating orbits in this potential}

Once you have a potential calculated, you should be able to integrate
orbits through it according to the ordinary differential equation: 
\begin{eqnarray}
\dot{\vec{v}} &=& -\vec{\nabla}\phi \cr
\dot{\vec{x}} &=& \vec{v} 
\end{eqnarray}
You can use Runge-Kutta. 

\begin{itemize}
\item Try starting a particle along a circular orbit with the right
  velocity in a spherical potential. Does it stay on a circular orbit?
\item Try making the potential non-spherical. What happens? Under what
  conditions does the orbit stay spherical? 
\end{itemize}

\end{document}
