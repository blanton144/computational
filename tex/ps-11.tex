\documentclass[11pt, preprint]{aastex}

\newcommand{\todo}[1]{{\bf #1}}

\newif\ifanswers


\begin{document}

\title{\bf Computational Physics / PHYS-UA 210 / Problem Set \#11
\\ Due November 22, 2019}

You {\it must} label all axes of all plots, including giving the {\it
  units}!!

\begin{enumerate}
  \item Exercise 8.3 of Newman.

\item Write a routine that integrates the equations for projectile
  motion:
  \begin{equation}
    \frac{\dd{^2\vec{x}}}{\dd{t}^2} = - g {\hat x}_1 - \alpha
    \dot{\vec{x}}^2,
  \end{equation}
  where ${\hat x}_1$ is the vertical direction.
  These are appropriate for, say, a golf ball. 
  The initial conditions should be that the object is launched at some
  angle $\theta$ from the horizontal at some initial speed in the
  $x_0$-$x_1$ plane. Integrate until the object hits the ground
  again. Use a Runge-Kutta method
  from {\tt scipy} to solve this problem and write a routine that
  finds where the ball hits the ground again.
\item Use Brent's method (either yours or {\tt scipy}'s) to optimize
  the angle $\theta$ to get the longest distance.

  \item Exercise 8.10 of Newman.
\end{enumerate}

\end{document}
