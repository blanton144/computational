Partial differential equations (PDEs) are distinguished from ordinary
differential equations by the presence of more than one independent
variable, which may be time or space. Generally speaking, they are
applicable to continuous systems: fluids, electromagnetic fields,
gravity, general relativity, etc. 

There are several types of PDEs, which can be classified as follows:
\begin{itemize}
\item Hyperbolic: second-order derivatives, with opposite signs, like
  the wave equation:
  \begin{equation}
\frac{\partial^2 u}{\partial t^2} - v^2 \frac{\partial^2 u}{\partial
  x^2} = 0
  \end{equation}
\item Elliptic: second-order derivatives, with like signs, like the
  Laplace equation:
  \begin{equation}
\frac{\partial^2 u}{\partial x^2} + \frac{\partial^2 u}{\partial
  y^2} = 0
  \end{equation}
\item Parabolic: first-order and second-order mixed, like the
  diffusion equation:
  \begin{equation}
\frac{\partial u}{\partial t} - \frac{\partial}{\partial
  x}\left(D\frac{\partial u}{\partial x^2}\right) = 0
  \end{equation}
\end{itemize}

These sorts of equations are more complex than ODEs, mainly because
you cannot reduce the equations to a single derivative of the state
variable with the single independent variable. Furthermore, you cannot
as easily set the boundary conditions; at least, there are many more
ways to. 

The simplest way to see this is the case of the Laplace equation. To
solve Laplace for an electric potential within some region requires
setting the potential at the surface of that region (or its
derivative, or some combination). Clearly this is a bit
complicated. Also, it is not at all obvious whether a solution with
the desired boundary conditions is necessarily allowed by the
equations (with Laplace, this is not really a problem of course, but
it can be with more general equations).

The setting of the boundary conditions is an extremely important
feature of any PDE problem, and usually defines the basic watersheds
between numerical approaches. For example, you can imagine setting
$u(x=0, t)$ and using the wave or diffusion equations to integrate in
time.  These are {\it Cauchy} or {\it initial value} problems.
But for Laplace, you need to specify {\it boundary values}, either of
$u$ ({\it Dirichlet}) or its derivative ({\it Neumann}).

\section{Boundary value problems}

We will concern ourselves first with boundary value problems.
