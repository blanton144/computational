\documentclass[11pt, preprint]{aastex}

\newcommand{\todo}[1]{{\bf #1}}

\newif\ifanswers


\begin{document}

\title{\bf Computational Physics / PHYS-UA 210 / Problem Set \#6
\\ Due October 11, 2019 }

You {\it must} label all axes of all plots, including giving the {\it
  units}!!

\begin{enumerate}
\item Exercise 5.15 in Newman.
\item This problem will explore interpolation a little so you have
  some experience with it. We will interpolate values of the $\sin()$
  function. In each part below, you will interpolate $\sin()$ from a
  grid of known values at $N$ equally spaced points for $x$ between
  $0$ and $10\pi$ (inclusive).
\begin{enumerate}
\item First, use linear interpolation, writing this code
  yourself. Test your code for $N=20$, $N=40$, $N=80$, and
  $N=160$. Quantify the rms residuals of the interpolation relative to
  $\sin()$ within the range of the grid, as a function of $N$.
\item Second, go ahead and utilize the {\tt interp1d} class in
  {\tt scipy.interpolate} to interpolate. Test the {\tt slinear}, {\tt
    quadratic}, and {\tt cubic} methods in the same way as above.
\item Third, add a little bit of noise to the values of $\sin()$ that
  you interpolate between; use Gaussian noise with a standard
  deviation of 0.1. Show some examples of how the interpolation
  behaves.
\end{enumerate}
  
\end{enumerate}

\end{document}
