\documentclass[11pt, preprint]{aastex}

\newcommand{\todo}[1]{{\bf #1}}

\newif\ifanswers


\begin{document}

\title{\bf Computational Physics / PHYS-UA 210 / Problem Set \#6
\\ Due October 20, 2017 }

You {\it must} label all axes of all plots, including giving the {\it
  units}!!

This homework focuses on fitting a linear model to a data set.
(Please note as an important point: when we say ``fitting a linear
model'' it means ``fitting a model whose predictions vary linearly
with its parameters,'' not ``fitting y vs. x with a line.'').

\begin{enumerate}
\item Generate a set of ``random'' $(x,y)$ data with constant noise,
  using $x$ in the range from 0 to 1, and with $y$ determined by an
  8th order polynomial:
  \begin{equation}
 y_i = \left[\sum_{j=0}^8 \alpha_j (x-0.5)^j\right] +
 \mathrm{~Gaussian~noise}
  \end{equation}
  Choose reasonable $\alpha_j$, and reasonable Gaussian noise
  (i.e. noticeable but not much larger than the features in your
  polynomial. 
\item For some (possibly different) set of coefficients,
  $\beta_j$, sum-squared residuals of the model $\hat y$ are:
  \begin{equation}
    S = \sum_i (y_i - {\hat y}_i)^2 = \sum_i \left[y_i - \sum_j
      {\beta}_j (x_i - 0.5)^j\right]^2
  \end{equation}
  which can be written as:
  \begin{equation}
    S = \left|\mat{A}\cdot \vec{\beta} - \vec{y} \right|^2
  \end{equation}
  Construct the matrix $\mat{A}$ given your random $\vec{x}$.
\item Use SVD to find the $\vec{\beta}$ that minimizes $S$. This is
  the linear least-squares estimate of $\vec{\alpha}$. Compare your
  model for $y$ with the $y_i$ values and with the original, correct
  $y$.  Try using different numbers of random draws: 6, 8, 32, 128.
  Compare including Gaussian noise to not adding any noise.
\item Compare using SVD to solving the ``normal equations.'' 
  The normal equations result from finding where:
  \begin{equation}
    \frac{\partial S}{\partial \vec{\beta}} = 0
  \end{equation}
  and yield the equation:
  \begin{equation}
    \left(\mat{A}^T\cdot \mat{A} \right)\cdot \vec{\beta} =
    \mat{A}^T\cdot\vec{y}
  \end{equation}
  This matrix equation can be solved by inverting $\mat{A}^T\cdot
  \mat{A}$, which is $N\times N$. E.g. you can just use the {\tt
    numpy.linalg.inv} implementation of LU decomposition. Try this
  technique for the examples you used SVD on and describe any
  differences you see.
\end{enumerate}

\end{document}
