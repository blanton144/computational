\documentclass[11pt, preprint]{aastex}

\newcommand{\todo}[1]{{\bf #1}}

\newif\ifanswers


\begin{document}

\title{\bf Computational Physics Project / 2D Ising Models}

This project focuses on the simulation through Monte Carlo methods of
2D Ising models for magnetization.

These model a magnetic material as a 2D lattice of atoms whose
magnetic moments can either be up or down. These are thermodynamic
systems, so depending on the temperature, there is more or less random
flipping of these moments.

Each neighboring pair of atoms (labeled $i$ and $j$, where $j$ is a
neighbor in $x$ or in $y$, but not along the diagonal) contributes to
the total energy of the system an amount:
\begin{equation}
  E = - J s_{i} s_{j}
\end{equation}
Aligned moments are lower energy, therefore are the preferred state.

Since the system is thermodynamic, the spins will sometimes flip
randomly. Flips that happen to be favorable happen at a constant rate,
but those that are unfavorable do so at a reduced rate; reduced
specifically by the Boltzman factor:
\begin{equation}
  R = \exp\left(-\Delta E /k T\right)
\end{equation}

We are interested in what happens to the magnetization of a system as
a function of temperature.  The magnetization can be defined as :
\begin{equation}
M = \sum_i s_i
\end{equation}

\begin{itemize}
\item Explain in words why you should find $M\sim 0$ for sufficiently
  high $T$, and find an extremal $M$ (all spins aligned) for
  sufficiently low $T$. 
\item Write into code the Metropolis algorithm found in Section 17.4
  of Landau. Note that you do {\it not} need to recalculate the full
  energy of the system at each trial, since most of it is unchanged
  (i.e. you only need the terms involving the lattice atom you
  picked).  Think about the scaling of the problem. Does your code
  need to carry around $J$, $k$, and $T$? Use periodic boundary
  conditions.
\item Now initialize your system to start with a fully magnetized (all
  spins aligned, with $s_i = 1$) system, at $T=0$. Raise the
  temperature in small steps, and after each step run the Metropolis
  algorithm for long enough that the energy and magnetization seem to
  have converged.
\item Plot the magnetization and energy as a function of
  temperature. Compare to the analytic solution found in Section
  17.3.1 of Landau (this will indicate to you how far up in $T$ you
  need to go).
\item What is the physical meaning of the slope of the energy versus
  temperature?  Can you describe in words what this slope means in
  terms of how much work it is to raise the temperature?
\item If you add a term to $E$ that accounts for an external magnetic
  field:
  \begin{equation}
    -H \sum_i s_i
  \end{equation}
  how does this change the behavior of the system when you start with
  $T=0$ and raise the temperature?  Try both $H$ positive and
  negative.
\item Try a few runs where you start with random spin orientations and
  $T\gg T_c$, and cool the system down. What do you notice about the
  final result? What if you include a non-zero $H$?
\end{itemize}

\end{document}
