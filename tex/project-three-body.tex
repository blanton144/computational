\documentclass[11pt, preprint]{aastex}

\newcommand{\todo}[1]{{\bf #1}}

\newif\ifanswers


\begin{document}

\title{\bf Computational Physics Project / Three-body problem}

This project involves integrating gravitational orbits of the sort
that you might find in planetary or stellar systems. 

The general equations governing these systems are simple:
\begin{eqnarray}
\dot{\vec{v}_i} &=& - \sum_{j \ne i} \frac{Gm_j}{r_{ij}^2} \cr
\dot{\vec{x}_i} &=& \vec{v}_i
\end{eqnarray}

For $N=2$, this is a classic Keplerian problem with known solutions.
A good introduction to this is given in the textbook {\it Principles
  of Astrophysics}, by Charles Keeton. Specifically, if $r$ is the
distance between the two particles, and $\phi$ is an angle in the
plane of motion measured with respect to the apocenter of the orbit,
then:
\begin{equation}
r = \frac{a(1-e^2)}{1+e \cos\phi}
\end{equation}
where $a$ is the semimajor axis of the orbit, and $e$ is the
eccentricity. The period of the orbit is:
\begin{equation}
P = \sqrt{\frac{4\pi^2 a^3}{G(m_1 + m_2)}}
\end{equation}
and the angle $\phi$ and time $t$ are related by:
\begin{equation}
  \frac{t}{P} = \frac{1}{2\pi}\left\{
    2\tan^{-1}\left[\left(\frac{1-e}{1+e}\right)^{1/2}
      \tan\frac{\phi}{2}\right]
    - \frac{e(1-e^2)^{1/2} \sin\phi}{1+e\cos\phi}\right\}
\end{equation}

\section{Testing the two-body problem}

In the first section of this project, you will write a general program
to integrate orbits, and test it on the two body problem.

\begin{itemize}
\item Write a module with an integrator using the techniques in Landau
  Chapter 8 that will solve the general system of equations for $N$
  bodies.
\item Test the integrator on the two body problem. Use initial
  conditions for which the system is bound. Assess the period of the
  orbit, how well it approximates an ellipse, and showing that it
  (approximately) closes. Use both circular and elliptical cases.
\end{itemize}

\section{Exploring the three-body problem}

Now you should be able explore how three-body systems behave. 

\begin{itemize}
\item Start with two large, equal mass particles, and one much smaller
  mass particle. Start the equal mass particles on circular orbits,
  and verify sure that they stay on those orbits. Then explore a
  couple of different initial conditions for the third particle.
\item Try the Lagrange Points (look the description up on Wikipedia)
  and put the small particle at those locations, with an angular
  velocity the same as the orbital angular velocity of the large mass
  particles. What happens?
\item Make one large mass particle, one somewhat smaller mass particle
  (like a factor of 100) in orbit around the large mass particle, and
  then a third particle. Try different initial conditions for the
  third particle, including some which are close to the orbit of the
  second particle. Show what happens as the orbits get close together.
\item Make three similar mass particles and follow their orbits. Can
  you find initial conditions which are stable? (Do not spend a ton of
  time trying).
\end{itemize}

\end{document}
