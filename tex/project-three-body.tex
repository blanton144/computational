\documentclass[11pt, preprint]{aastex}

\newcommand{\todo}[1]{{\bf #1}}

\newif\ifanswers


\begin{document}

\title{\bf Computational Physics Project / Three-body problem}

This project involves integrating gravitational orbits of the sort
that you might find in planetary or stellar systems. 

The general equations governing these systems are simple:
\begin{eqnarray}
\dot{\vec{v}_i} &=& - \sum_{j \ne i} \frac{Gm_j}{r_{ij}^2} {\hat r}_{ij} \cr
\dot{\vec{x}_i} &=& \vec{v}_i
\end{eqnarray}

For $N=2$, this is a classic Keplerian problem with known solutions.
A good introduction to this is given in the textbook {\it Principles
  of Astrophysics}, by Charles Keeton. Specifically, if $r$ is the
distance between the two particles, and $\phi$ is an angle in the
plane of motion measured with respect to the apocenter of the orbit,
then:
\begin{equation}
r = \frac{a(1-e^2)}{1+e \cos\phi}
\end{equation}
where $a$ is the semimajor axis of the orbit, and $e$ is the
eccentricity. The period of the orbit is:
\begin{equation}
P = \sqrt{\frac{4\pi^2 a^3}{G(m_1 + m_2)}}
\end{equation}
and the angle $\phi$ and time $t$ are related by:
\begin{equation}
  \frac{t}{P} = \frac{1}{2\pi}\left\{
    2\tan^{-1}\left[\left(\frac{1-e}{1+e}\right)^{1/2}
      \tan\frac{\phi}{2}\right]
    - \frac{e(1-e^2)^{1/2} \sin\phi}{1+e\cos\phi}\right\}
\end{equation}

\section{Prep work}

First you should do some prep work to understand the problem and test
your tools.

\begin{itemize}
\item Define a new unitless set of variables for position, mass,
  velocity, and time, by scaling out the length scale, time scale,
  total mass scale, and $G$. There should be a single combination of
  these overall scale values that characterizes the expression. You
  have the freedom to choose to set this combination to unity. You
  should perform your numerical analysis in these variables; your
  numerical solutions can then be scaled to different total mass and
  lengths by keeping this combination fixed.
\item Using minimization and/or root-finding techniques, write a
  program that takes as input a function that returns a radius given
  the time, and finds the pericenter positions in the orbit and
  determines the period. Use the analytic solution for the two-body
  problem above to verify your work.
\item The equations for the problem can be evaluated in a rotating
  frame. If the rotation is characterized by a constant angular
  velocity $\vec{\Omega}$, then the equations in the rotating frame
  for $\vec{x}_i'$ and $\vec{v}_i'$ in the rotating frame are:
  \begin{eqnarray}
    \dot{\vec{v}_i'} &=& - \sum_{j \ne i} \frac{Gm_j}{r_{ij}^2} {\hat
      r}_{ij}
    -2 \vec{\Omega}\times\vec{v_i}' -
    \vec{\Omega}\times\vec{\Omega}\times\vec{x}_i'  \cr
    \dot{\vec{x}_i'} &=& \vec{v}_i'
  \end{eqnarray}
  where $\dot{\vec{x}'} = \dot{\vec{x}} - \vec\Omega \times \vec{x}$.
  Consider a two body system in circular orbit; this specifies a
  constant $\Omega$. If there is a third test body in the system with
  $\vec{v}'=0$, then you can think of the forces acting on it as the
  derivative of an effective potential that accounts for the
  gravitational potential Coriolis term together (the first and third
  terms on the right hand side above). Write down that effective
  potential. Then plot the effective potential in the plane of the
  orbit of the two massive bodies. Identify the stationary
  points. These are called {\it Lagrange Points}. What is special
  about them?
\end{itemize}

\section{Creating the integrator}

In this stage, you will write a general program to integrate orbits,
and test it on the two body problem.
  
\begin{itemize}
\item Write a module with an integrator using the techniques in 
  Chapter 8 that will solve the general system of equations for $N$
  bodies.
\item Test the integrator on the two body problem. Use initial
  conditions for which the system is bound. Use the tools you
  developed above to test the period of the orbit as a function of
  mass and semi-major axis. Test both circular and elliptical cases,
  and unequal mass cases.
\end{itemize}

\section{Exploring the three-body problem}

Now you should be able explore how three-body systems behave. 

\begin{itemize}
\item Start with two large, equal mass particles, and one much smaller
  mass particle. Start the equal mass particles on circular orbits,
  and verify sure that they stay on those orbits. 
\item Test the Lagrange Points discussed above.  Put the small
  particle at those locations, with an angular velocity the same as
  the orbital angular velocity of the large mass particles. Does the
  particle do what it should?
\item Perform the same test, but work in the rotating frame by using
  the variables $\vec{x}'$ and $\vec{v}'$, using the $\Omega$ value
  appropriate for the two massive particles alone. 
\item Make one large mass particle and two somewhat smaller mass
  particles (like a factor of 100). Put one of the smaller mass
  particles in a stable circular two-body orbit around the large mass
  particle. Try many different initial conditions for the third
  particle that correspond to circular orbits with a range of
  different radii. When does the system remain stable and when not?
\item Make three similar mass particles and follow their orbits. Can
  you find initial conditions which are stable? (Do not spend a ton of
  time trying).
\end{itemize}

\section{Bonus ideas}

\begin{itemize}
\item Test integrators of different orders on the two-body problem and
  test how solution accuracy relates to calculation time for each.
\end{itemize}

\end{document}
