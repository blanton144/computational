\documentclass[11pt, preprint]{aastex}

\newcommand{\todo}[1]{{\bf #1}}

\newif\ifanswers


\begin{document}

\title{\bf Computational Physics / PHYS-UA 210 / Problem Set \#8
\\ Due November 10, 2017 }

You {\it must} label all axes of all plots, including giving the {\it
  units}!!

\begin{enumerate} 
\item Implement Brent's method for minimization of a one-dimensional
  function. Test it on a fourth-order function. Compare your
  performance to the {\tt scipy} implementation.
\item Write a routine that integrates the equations for projectile
  motion:
  \begin{equation}
    \frac{\dd{^2\vec{x}}}{\dd{t}^2} = - g {\hat x}_1 - \alpha
    \dot{\vec{x}}^2
  \end{equation}
  These are appropriate for, say, a golf ball. 
  The initial conditions should be that the object is launched at some
  angle $\theta$ from the horizontal at some initial speed in the
  $x_0$-$x_1$ plane. Integrate until the object hits the ground
  again. Implement Euler's method, and also use a Runge-Kutta method
  from {\tt scipy}. Compare their convergence with time steps.
\item Use Brent's method (either yours or {\tt scipy}'s) to optimize
  the angle $\theta$ to get the longest distance.
\item Now consider that you are aiming for a specific point along the
  ${\hat x}_0$ axis, and that there is some wind in the ${\hat x}_2$
  direction, so the drag term becomes: 
  \begin{equation}
    - \alpha (\dot{\vec{x}} - w {\hat x_2})^2 
  \end{equation}
  Allow yourself the freedom to hit the golf ball in any direction
  $\theta$ from the horizontal, any angle $phi$ from the $x_0$ axis in
  the $x_0$-$x_2$ plane, and at any speed. You should be able to hit
  any given spot $D$ along the $x_0$ axis. Use the {\tt scipy}
  implementation of {\tt bfgs} to perform this optimization.
\end{enumerate} 


\end{document}
