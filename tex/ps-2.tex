\documentclass[11pt, preprint]{aastex}

\newcommand{\todo}[1]{{\bf #1}}

\newif\ifanswers


\begin{document}

\title{\bf Computational Physics / PHYS-UA 210 / Problem Set \#2
\\ Due September 13, 2019 }

You {\it must} label all axes of any plots, including giving the {\it
  units}!!

\begin{enumerate}

  \item What is the binary representation of the decimal integer 121? 

  \item Write a function in Python which calculates and returns the
    sum of a NumPy array of numbers by executing a loop which performs
    a set of additions. Note that you can achieve the same
    functionality with the {\tt sum()} method of NumPy array
    objects. Compare the speed of your function vs. {\tt sum()} using
    {\tt \%timeit}. 

  \item Exercise 2.9 of Newman. Note that the physical constants drop
    out so you do not need to worry about them (whenever possible you
    should seek to remove physical constants from the innards of your
    computations!). Also, you should try to solve this without using
    an explicit {\tt for} loop.

  \item Exercise 4.2 of Newman.

\end{enumerate}

\end{document}
