\documentclass[11pt, preprint]{aastex}

\newcommand{\todo}[1]{{\bf #1}}

\newif\ifanswers


\begin{document}

\title{\bf Computational Physics / PHYS-UA 210 / Problem Set \#2
\\ Due September 13, 2019 }

You {\it must} label all axes of any plots, including giving the {\it
  units}!!

\begin{enumerate}

  \item What is the binary representation of the decimal integer 121? 

  \item Exercise 2.9 of Newman. Note that the physical constants drop
    out so you do not need to worry about them (whenever possible you
    should seek to remove physical constants from the innards of your
    computations!).  Write two versions of the code, one which uses a
    {\tt for} loop and one which does not.  Use {\tt \%timeit} to
    determine which is faster.

  \item Exercise 3.7 of Newman.  Note that you can use a NumPy array
    to perform the iterations for each value of $c$ all at once, which
    will be much faster than using a {\tt for} loop over $c$.

  \item Exercise 4.2 of Newman.

\end{enumerate}

\end{document}
