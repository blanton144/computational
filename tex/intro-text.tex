The goals of Computational Physics are to learn the principles of
using computers to perform calculations in physics. These principles
include how numbers are represented, and the basic tools for
calculating special functions, random numbers, linear algebra,
interpolation, root-finding and optimization, differentiation, 
integration, and spectral analysis.

It is helpful to understand what we will not be doing:
\begin{itemize}
\item We will not here be considering explicitly problems in physics
  data analysis, but many of the techniques we learn here will be
  applicable in that context.
\item I will not be teaching you the principles of software
  engineering. To the extent possible, I will emphasize the importance
  of documentation, modularity, validation, and version control in
  writing stable, maintainable code. But do not underestimate the
  level of experience, discipline, and effort that good software
  engineering requires, and respect it when you see it.
\end{itemize}

I will lecture and demonstrate in class, and there will be homeworks
most of the semester. There will also be a set of large-ish
projects. You will work on these in pairs, and you will let me know
which teams you are on by Sept 26, at which point I will assign
topics.

We must choose a language to work in. There are many available
computer languages. Useful ones are C, C++, Fortran, Java, Python, and
Julia. There are other field-specific languages in use as well
(e.g. in astronomy there are IRAF and IDL).

Here we will use Python, mostly because it is a modern language of
broad applicability, which also allows easy visualization within the
language itself. It has limitations: you would probably not want to
write a very computationally heavy simulation in Python, but more
likely directly in C++. But for many calculations it works fine and is
convenient, and in any case in this course we will concern ourselves
mostly with the concepts of computational physics rather than aiming
at the most efficient possible implementations.

In addition, you will use a specific, quite useful 

To install Python on your computer and enough of its packages to
function, I recommend the Anaconda Python 3 distribution. Other
packages can be installed as described in the book. 

Unfortunately, for a class like this we have to go through the basics
of getting everybody set up on their computers and using the tools of
the course. So there is some boring stuff to get through before we get
to the fun stuff. 

To do so, I want to describe what each of you will need in your
environment. The end point requirements are:
\begin{itemize}
\item git
\item Python
\item Jupyter
\end{itemize}
and as long as those work on your system you should be OK. In
principle, these should all be able to work on Windows, Linux, or Mac
OS X. Probably Linux and Mac OS X are the easiest though.

For {\tt git}, you will also need to create a GitHub account. You
should create this account, and then start a GitHub repository
specifically for this course. I recommend that you structure your repo
as follows:
\begin{verbatim}
 cp210/
  scr/
  tex/ 
  homeworks/
   ps-[n].ipynb
\end{verbatim}
where you can stored scratch work you want to save in {\tt scr}, any
work within Latex within {\tt tex}, and put your homework for the TA
to find them in {\tt homeworks}, named as described. 




