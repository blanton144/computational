\documentclass[11pt, preprint]{aastex}
\usepackage{hyperref} 
\usepackage{rotating}
 
\setlength{\footnotesep}{9.6pt}

\newcounter{thefigs}
\newcommand{\fignum}{\arabic{thefigs}}

\newcounter{thetabs}
\newcommand{\tabnum}{\arabic{thetabs}}

\newcounter{address}

\begin{document}

\title{\bf Computational Physics / PHYS-UA 210}
~
~

\noindent This course teaches computational physics at a level
appropriate for undergraduate physics majors.  Classes meet Tuesday
and Thursday 12:30am to 1:45pm, in Room 1067 of 726 Broadway.  The
textbook is {\it Computational Physics}, by Mark Newman. I will also
ask you sometimes to look online at the Python Data Science Handbook
(PDSH) by Jake Van Der Plas.

\noindent If you have never programmed in Python before then Chapter 2
of the book will require your special attention. There are also many
online resources for learning the basics of Python. I can recommend
Software Carpentry.

\noindent Prof. Blanton's office is Room 941 of 726 Broadway, and his
email is {\tt blanton@nyu.edu}.  Office hours are Tuesdays 11:00am to
12:15pm.

\noindent The teaching assistant is Aditya Hardikar ({\tt
  avh292@nyu.edu}) Recitation is Thursday, 5pm--6:15pm, also in Room
1067 of 726 Broadway. This time will primarily consist in working on
homework assignments.

\noindent The class will be participatory. Please read the assignments
          {\it before} attending class; you will be expected at
          certain points to follow along with calculations on your
          computer.

\noindent There will be no exams in this course, but there will be a
pretty heavy load of assignments:
\begin{itemize}
\item You will complete roughly weekly homeworks. You may consult with
  each other about the homeworks, but you must write your own
  individual code and report. This report will be in the form of
  rendered Jupyter notebooks and/or \LaTeX\ documents, submitted to
  the TA in a manner they prescribe.
\item The second major assignment is a large project performed in
  groups of two or three students each, culminating in a presentation
  in December. You will hand in a report written using \LaTeX, a
  standard format for physics research that you might as well become
  familiar with (the homeworks will introduce this format). Two
  intermediate drafts are due of this large project, which I will
  comment on. The projects are designed so that you can complete parts
  of them during the course of the semester based on
  material already covered. 
\end{itemize}

\noindent Grades are based on problem sets (65\%), the large project
and presentation (25\%), and class participation (10\%).

\noindent The classes will proceed as follows (subject to revision!).
The problem sets will be due on each Friday of the indicated weeks.

\baselineskip 0pt
\begin{table}[h!]
\footnotesize
\begin{tabular}{|c|c|c|c|}
\hline
{\it Date} & {\it Topic} & {\it Reading} & {\it Problem Sets} \cr  
\hline 
2019-09-03 (T) & Numbers on computers  & Ch.~1, 2, 3 & \cr
2019-09-05 (R) & Arrays                & 
\href{https://github.com/jakevdp/PythonDataScienceHandbook/tree/de0cc6bd317012d50ab3dd06e3cf4e256de1973f/notebooks}{PDSH,
  Ch. 1 \& 2} & PS\#1 \cr
2019-09-10 (T) & Numerics           & Ch.~4 & \cr
2019-09-12 (R) & Random Numbers     & Ch.~10.1 & PS\#2 \cr
2019-09-17 (T) & Random Numbers     & Ch.~10.2 & Teams Determined \cr
2019-09-19 (R) & Integration        & Ch.~5.1--5.3 & PS\#3 \cr
2019-09-24 (T) & Integration        & Ch.~5.4--5.6 & --- \cr
2019-09-26 (R) & Integration        & Ch.~5.7--5.9 & PS\#4 \cr
2019-10-01 (T) & Differentiation    & Ch.~5.10--5.11 & \cr
2019-10-03 (R) & Linear Algebra     & Ch.~6.1 & PS\#5 \cr
2019-10-08 (T) & Linear Algebra     & Ch.~6.1 & \cr
2019-10-10 (R) & Eigensystems       & Ch.~6.2 & PS\#6 \cr
2019-10-15 (T) & {\bf Legislative Day, no class}       & Ch.~6.2 & \cr
2019-10-17 (R) & Eigensystems       & Ch.~6.2 & PS\#7 \cr
2019-10-22 (T) & Eigensystems       & Ch.~6.2 & \cr
2019-10-24 (R) & Root-finding       & Ch.~6.3 & PS\#8 \cr
2019-10-29 (T) & Fourier Analysis   & Ch.~7.1--7.2 & \cr
2019-10-31 (R) & Functional Programming & --- & PS\#9 \cr
2019-11-05 (T) & Fourier Analysis   & Ch.~7.3--7.4 & \cr
2019-11-07 (R) & Minimization       & Ch.~6.4 & Project draft due\cr
2019-11-12 (T) & Minimization       & Ch.~6.4 & \cr
2019-11-14 (R) & Ordinary DEs        & Ch.~8.1--8.3 & PS\#10 \cr
2019-11-19 (T) & Ordinary DEs        & Ch.~8.4--8.5 & \cr
2019-11-21 (R) & Ordinary DEs        & Ch.~8.6 & PS\#11 \cr
2019-11-26 (T) & Partial DEs        & Ch.~9.1--9.2 & \cr
2019-11-28 (R) & {\bf Thanksgiving, no class} & & \cr
2019-12-03 (T) & Partial DEs        & Ch.~9.3 & \cr
2019-12-05 (R) & Partial DEs        & Ch.~9.3  & Final project due\cr
2019-12-10 (T) & Project presentations & --- & \cr
2019-12-12 (R) & Project presentations & --- & PS\#12\cr
\hline
\end{tabular}
\end{table}

\end{document}

