\documentclass[11pt, preprint]{aastex}
\usepackage{hyperref} 
\usepackage{rotating}
 
\setlength{\footnotesep}{9.6pt}

\newcounter{thefigs}
\newcommand{\fignum}{\arabic{thefigs}}

\newcounter{thetabs}
\newcommand{\tabnum}{\arabic{thetabs}}

\newcounter{address}

\begin{document}

\title{\bf Computational Physics / PHYS-UA 210}
~
~

\noindent This course teaches computational physics at a level
appropriate for undergraduate physics majors.  Classes meet Tuesday
and Thursday 12:30am to 1:45pm, in 12 Waverly Place, L113 {\bf
  UPDATE}.  The textbook is {\it Computational Physics}, by Mark
Newman. I will also ask you sometimes to look online at the Python
Data Science Handbook (PDSH) by Jake Van Der Plas.

\noindent Prof. Blanton's office is Room 941 of 726 Broadway, and his
email is {\tt blanton@nyu.edu}.  Office hours are Wednesday 2:00pm to
3:15pm.

\noindent The teaching assistant is Shengqi Yang (sy1823@nyu.edu) {\bf
  UPDATE}. Recitation is Wednesday 5:00pm to 6:15pm {\bf UPDATE}. This
time will primarily consist in working on homework assignments.

\noindent The class will be participatory. Please read the assignments
          {\it before} attending class; you will be expected at
          certain points to follow along with calculations on your
          computer.

\noindent There will be no exams in this course, but there will be a
pretty heavy load of assignments:
\begin{itemize}
\item You will complete weekly homeworks. You may consult with each
  other about the homeworks, but you must write your own individual
  code and report. This report will be in the form of rendered Jupyter
  notebooks and/or \LaTeX\ documents, submitted to Yang {\bf UPDATE}
  in a manner she prescribes.
\item The second major assignment is a large project performed in
  groups of two or three students each, culminating in a presentation
  in December. You will hand in a report written using \LaTeX, a
  standard format for physics research that you might as well become
  familiar with (the homeworks will introduce this format). A draft
  report will be due by November 17 {\bf UPDATE}.
\end{itemize}

\noindent Grades are based on problem sets (65\%), a project with
results presented to the class (25\%), and class participation (10\%).

\noindent The classes will proceed as follows (subject to revision!).
The problem sets will be due on each Friday of the indicated weeks.

\baselineskip 0pt
\begin{table}[h!]
\footnotesize
\begin{tabular}{|c|c|c|c|}
\hline
{\it Date} & {\it Topic} & {\it Reading} & {\it Problem Sets} \cr  
\hline 
2017-09-05 (T) & Numbers on computers  & Ch.~1, 2, 3 & \cr
2017-09-07 (R) & Arrays                & 
\href{https://github.com/jakevdp/PythonDataScienceHandbook/tree/de0cc6bd317012d50ab3dd06e3cf4e256de1973f/notebooks}{PDSH, Ch. 1 \& 2} & \cr
2017-09-12 (T) & Numerics           & Ch.~4 & \cr
2017-09-14 (R) & Numerics           & Ch.~4 & PS\#1 \cr
2017-09-19 (T) & Random Numbers     & Ch.~10 & \cr
2017-09-21 (R) & Random Numbers     & Ch.~10 & PS\#2 \cr
2017-09-26 (T) & Integration        & Ch.~5.1--5.3 & Teams declared \cr
2017-09-28 (R) & Integration        & Ch.~5.4--5.6 & PS\#3 \cr
2017-10-03 (T) & Integration        & Ch.~5.7--5.9 & \cr
2017-10-05 (R) & Differentiation    & Ch.~5.10--5.11 & PS\#4 \cr
2017-10-10 (T) & Linear Algebra     & Ch.~6.1 & \cr
2017-10-12 (R) & Linear Algebra     & Ch.~6.1 & PS\#5 \cr
2017-10-17 (T) & Eigensystems       & Ch.~6.2 & \cr
2017-10-19 (R) & Eigensystems       & Ch.~6.2 & PS\#6 \cr
2017-10-24 (T) & Root-finding       & Ch.~6.3 & \cr
2017-10-26 (R) & Minimization       & Ch.~6.4 & PS\#7 \cr
2017-10-31 (T) & Fourier Analysis   & Ch.~7 & \cr
2017-11-02 (R) & Fourier Analysis   & Ch.~7 & PS\#8  \cr
2017-11-07 (T) & Ordinary DEs       & Ch.~8 & \cr
2017-11-09 (R) & Ordinary DEs       & Ch.~8 & --- \cr
2017-11-14 (T) & Partial DEs        & Ch.~9 & Presentations begin\cr
2017-11-16 (R) & Partial DEs        & Ch.~9 & Draft report due\cr
2017-11-21 (T) & Partial DEs        & Ch.~9 & \cr
2017-11-23 (R) & {\bf Thanksgiving, no class} & & \cr
2017-11-28 (T) & Diffusion          & --- & \cr
2017-11-30 (R) & Nonlinear dynamics & --- & \cr
2017-12-05 (T) & --- & ---  & \cr
2017-12-07 (R) & --- & --- & \cr
2017-12-12 (T) & {\bf Legislative Day, no class} & & \cr
2017-12-14 (R) & --- & --- & Final report due\cr
\hline
\end{tabular}
\end{table}

\end{document}

