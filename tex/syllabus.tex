\documentclass[11pt, preprint]{aastex}
\usepackage{hyperref} 
\usepackage{rotating}
 
\setlength{\footnotesep}{9.6pt}

\newcounter{thefigs}
\newcommand{\fignum}{\arabic{thefigs}}

\newcounter{thetabs}
\newcommand{\tabnum}{\arabic{thetabs}}

\newcounter{address}

\begin{document}

\title{\bf Computational Physics / PHYS-UA 210}
~
~

\noindent This course teaches computational physics at a level
appropriate for undergraduate physics majors.  Classes meet Tuesday
and Thursday 12:30am to 1:45pm, in 12 Waverly Place, L113.  The
textbook is {\it Computational Physics}, by Rubin Landau. I will also
ask you sometimes to look online at the Python Data Science Handbook
(PDSH) by Jake Van Der Plas.

\noindent Prof. Blanton's office is Room 941 of 726 Broadway, and his
email is {\tt blanton@nyu.edu}.  Office hours are Wednesday 2:00pm to
3:15pm.

\noindent The teaching assistant is Shengqi Yang
(sy1823@nyu.edu). Recitation is Wednesday 5:00pm to 6:15pm. This time
will primarily consist in working on homework assignments.

\noindent The class will be participatory. Please read the assignments
before attending class; you will be expected at certain points to
follow along with calculations on your computer.

\noindent There will be no exams in this course, but there will be a
pretty heavy load of assignments:
\begin{itemize}
\item You will complete weekly homeworks until the beginning of
  November, and a final one due the last week of classes. You may
  consult with each other about the homeworks, but you must write your
  own individual code and report. This report will be in the form of
  rendered Jupyter notebooks or Latex documents, submitted to Yang in
  a manner she prescribes.
\item The second major assignment is a large-ish project performed in
  groups of two students each, culminating in a presentation in
  November or December. You will hand in a report written using Latex,
  a standard format for physics research that you might as well become
  familiar with (the homeworks will introduce this format). A draft
  report will be due by November 17.
\end{itemize}

\noindent Grades are based on problem sets (65\%), a project with
results presented to the class (25\%), and class participation (10\%).

\noindent The classes will proceed as follows (subject to revision!).
The problem sets will be due on each Friday of the indicated weeks.

\baselineskip 0pt
\begin{table}[h!]
\footnotesize
\begin{tabular}{|c|c|c|c|}
\hline
{\it Date} & {\it Topic} & {\it Reading} & {\it Problem Sets} \cr  
\hline 
2017-09-05 (T) & Numbers on computers  & Ch.~1 \& 2 & \cr
2017-09-07 (R) & Arrays                & 
\href{https://github.com/jakevdp/PythonDataScienceHandbook/tree/de0cc6bd317012d50ab3dd06e3cf4e256de1973f/notebooks}{PDSH, Ch. 1 \& 2} & \cr
2017-09-12 (T) & Numerics           & Ch.~3 & \cr
2017-09-14 (R) & Numerics           & Ch.~3 & PS\#1 \cr
2017-09-19 (T) & Random Numbers     & Ch.~4 & \cr
2017-09-21 (R) & Random Numbers     & Ch.~4 & PS\#2 \cr
2017-09-26 (T) & Differentiation    & Ch.~5 & Teams declared \cr
2017-09-28 (R) & Integration        & Ch.~5 & PS\#3 \cr
2017-10-03 (T) & Integration        & Ch.~5 & \cr
2017-10-05 (R) & Integration        & Ch.~6 & PS\#4 \cr
2017-10-10 (T) & Linear Algebra     & Ch.~6 & \cr
2017-10-12 (R) & Eigensystems       & Ch.~6 & PS\#5 \cr
2017-10-17 (T) & Root-finding       & Ch.~7 & \cr
2017-10-19 (R) & Minimization       & Ch.~7 & PS\#6 \cr
2017-10-24 (T) & Ordinary DEs       & Ch.~8 & \cr
2017-10-26 (R) & Ordinary DEs       & Ch.~8 & PS\#7 \cr
2017-10-31 (T) & Fourier Analysis   & Ch.~12 & \cr
2017-11-02 (R) & Fourier Analysis   & Ch.~12 & PS\#8  \cr
2017-11-07 (T) & Wavelet Analysis   & Ch.~13 & \cr
2017-11-09 (R) & Principal Components & Ch.~13 & Presentations begin \cr
2017-11-14 (T) & Partial DEs        & Ch.~19 & \cr
2017-11-16 (R) & Optimization       & Ch.~11 & Draft report due\cr
2017-11-21 (T) & Diffusion          & Ch.~20 & \cr
2017-11-23 (R) & {\bf Thanksgiving, no class} & & \cr
2017-11-28 (T) & Nonlinear dynamics & Ch.~15 & \cr
2017-11-30 (R) & Two-body gravity   & --- & \cr
2017-12-05 (T) & Three-body gravity & ---  & \cr
2017-12-07 (R) & $N$-body gravity   & --- & \cr
2017-12-12 (T) & {\bf Legislative Day, no class} & & \cr
2017-12-14 (R) & Parallel computing & --- & PS\#9 \cr
\hline
\end{tabular}
\end{table}

\end{document}

