\documentclass[11pt, preprint]{aastex}

\newcommand{\todo}[1]{{\bf #1}}

\newif\ifanswers


\begin{document}

\title{\bf Computational Physics Project / Solitons}

This project focuses on one-dimensional waves, as one might find in
shallow water in a narrow canal (once a practical situation, but even
now a useful idealized approximation). Much of the discussion here
comes from the computational physics textbook of Landau, P{\'a}ez, \&
Bordeianu (not {\it the} Landau).

The equation governing this situation is the Kortweg \& de Vries
(1895) equation:
\begin{equation}
\label{eq:soliton}
\frac{\partial u(x,t)}{\partial t} + \epsilon u(x,t) \frac{\partial
  u(x,t)}{\partial x} + \mu \frac{\partial^3 u(x,t)}{\partial x^3} = 0
\end{equation}
where $u$ is the height of the water, $x$ is the direction of
propagation, and $t$ is time. The second term is a nonlinear advection
term that makes a wave propagate; roughly speaking, the speed of
propagation increases with the height of the wave. The third term is a
dispersive term; roughly speaking, gradients in the height cause the
wave to spread out.

\section{Prep work}

Before trying to solve this problem numerically, it is worthwhile to
consider some related problems and analytic solutions. 

\begin{itemize}
\item First, consider the simpler problem:
  \begin{equation}
   \frac{\partial u(x,t)}{\partial t} + \epsilon \frac{\partial
     u(x,t)}{\partial x} = 0
  \end{equation}
This is the classic {\it advection} equation. Show that any function
of the form
   \begin{equation}
     u(x,t) = f(x- \epsilon t) 
   \end{equation}
is a solution. What is the interpretation of this form of solution?
What is the interpretation of $\epsilon$ in this case?
\item Given the previous result, make a qualitative prediction
  regarding what happens in the nonlinear, but non-dispersive, case,
  where $\mu = 0$ in Equation \ref{eq:soliton}.
\item It so happens that there are solutions of the form 
   \begin{equation}
     u(x,t) = f(x- c t) 
   \end{equation}
   for the full Equation \ref{eq:soliton}, but that unlike the
   advection equation, not {\it any} function of that form is a
   solution. Show that if $\xi = x - c t$, then these solutions must
   satisfy the ordinary differential equation:
   \begin{equation}
      - c \frac{\partial u}{\partial \xi} + \epsilon u \frac{\partial
        u}{\partial \xi} + \mu \frac{\partial^3 u}{\partial xi^3} = 0.
   \end{equation}
 \item Further show that a solution to this equation is:
   \begin{equation}
    u(x,t) = - \frac{c}{2} \sech^2\left[\frac{1}{2} \sqrt{c} \left(x-
      ct - \xi_0\right) \right],
   \end{equation}
   where $\xi_0$ is a constant representing an initial phase.  Plot
   this solution, which is known as a soliton. Can you interpret
   qualitatively why this solution exists?
\end{itemize}

\section{Designing the code}

We will use a finite difference technique. You will need to choose
$\Delta x$ and $\Delta t$. We will refer to steps in time as $j$ and
steps in space as $i$.

\begin{itemize}
\item What are the first-order central difference approximations for
  $\partial u/\partial t$ and $\partial u/\partial x$?
\item Expand $u(x,t)$ in a Taylor series in $x$ around $x_i$ to third
  order, at fixed time $t$.  For the four points $x_i - 2 \Delta x$,
  $x_i - \Delta x$, $x+\Delta x$ and $x+2 \Delta x$, you have the
  values $u_{i-2,j}$, $u_{i-1,j}$, $u_{i+1, j}$ and $u_{i+2, j}$.  You
  should be able to solve this set of equations for $\partial^3 u /
  \partial x^3$ at $x=x_i$.
\item Combine the above results into an evolution scheme for this set
  of equations, where you use where needed:
\begin{equation}
 u(x_i, t_j) = \frac{u_{i-1, j} + u_{i,j} + u_{i+1,j}}{3}
\end{equation}
\item Can you determine the relationship between $\Delta x$ and
  $\Delta t$ necessary for linear stability of this scheme?
\item To get started, you need two time steps. Write how you do this
  with a forward difference in time instead of a central difference.
\end{itemize}

The above equations do not handle the boundary conditions. Thus, in
the very first time step, $u_{1,2}$, $u_{2,2}$, $u_{N-1, 2}$, and
$u_{N, 2}$ are not determined. In the implementation below, you will
assume fixed values of $u_{1, j}$ and $u_N, j$ and a zero derivative
(i.e. for the purposes of the finite difference that $u_{0, j} =
u_{1,j}$ and $u_{N+1, j} = u_{N,j}$).

\section{Testing and running the code}

\begin{itemize}
\item Write an implementation of the finite difference solution to
  these equations.
\item Set $\mu = 0$, and make the second term linear
  (i.e. without the prefactor $u$). This is just the advection
  equation. Show that your code successfully advects any initial
  waveform. Use boundary conditions of zero, and make the grid big
  enough that your test stays away from the boundary.
\item Set $\mu = 0$, but use the nonlinear second term. This
  special case is known as {\it Burgers' equation}. Show what happens
  to an initially Gaussian wave and interpret it in the context of
  what you thought would happen in the prep work section. Use boundary
  conditions of zero again.
\item For non-zero $\mu$ and $\epsilon$, set initial conditions of a
  soliton wave. Show that it propagates as expected. Use boundary
  conditions of zero again.
\item For make initial conditions of two soliton waves, one taller
  than and ``behind'' the second one. What happens? Use boundary
  conditions of zero again.
\item Finally, try initial conditions of:
  \begin{equation}
    u(x, t=0) = \frac{1}{2} \left[1 -
      \tanh\left(\frac{x-x_0}{\sigma}\right)\right].
  \end{equation}
  Use boundary conditions of $u=1$ on the left and $u=0$ on the
  right. Watch what happens and interpret it.
\end{itemize}

\end{document}
